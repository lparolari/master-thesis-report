\documentclass{article}

\usepackage[utf8]{inputenc}
\usepackage{amsmath}
\usepackage{amsfonts}
\usepackage{bm}

\newcommand*{\todo}[1]{\footnote{\textbf{TODO:} {#1}}}
\newcommand*{\etal}[1]{\textit{et al.}}

% special letters denoting sets and algebras
\providecommand*{\Nset}{\mathbb{N}}  % naturals
\providecommand*{\Qset}{\mathbb{Q}}  % rationals
\providecommand*{\Zset}{\mathbb{Z}}  % integers
\providecommand*{\Rset}{\mathbb{R}}  % reals

% calligraphic alphabet
\newcommand*{\calA}{\ensuremath{\mathcal{A}}}
\newcommand*{\calB}{\ensuremath{\mathcal{B}}}
\newcommand*{\calC}{\ensuremath{\mathcal{C}}}
\newcommand*{\calD}{\ensuremath{\mathcal{D}}}
\newcommand*{\calE}{\ensuremath{\mathcal{E}}}
\newcommand*{\calF}{\ensuremath{\mathcal{F}}}
\newcommand*{\calG}{\ensuremath{\mathcal{G}}}
\newcommand*{\calH}{\ensuremath{\mathcal{H}}}
\newcommand*{\calI}{\ensuremath{\mathcal{I}}}
\newcommand*{\calJ}{\ensuremath{\mathcal{J}}}
\newcommand*{\calK}{\ensuremath{\mathcal{K}}}
\newcommand*{\calL}{\ensuremath{\mathcal{L}}}
\newcommand*{\calM}{\ensuremath{\mathcal{M}}}
\newcommand*{\calN}{\ensuremath{\mathcal{N}}}
\newcommand*{\calO}{\ensuremath{\mathcal{O}}}
\newcommand*{\calP}{\ensuremath{\mathcal{P}}}
\newcommand*{\calQ}{\ensuremath{\mathcal{Q}}}
\newcommand*{\calR}{\ensuremath{\mathcal{R}}}
\newcommand*{\calS}{\ensuremath{\mathcal{S}}}
\newcommand*{\calT}{\ensuremath{\mathcal{T}}}
\newcommand*{\calU}{\ensuremath{\mathcal{U}}}
\newcommand*{\calV}{\ensuremath{\mathcal{V}}}
\newcommand*{\calW}{\ensuremath{\mathcal{W}}}
\newcommand*{\calX}{\ensuremath{\mathcal{X}}}
\newcommand*{\calY}{\ensuremath{\mathcal{Y}}}
\newcommand*{\calZ}{\ensuremath{\mathcal{Z}}}

% bold letters
\newcommand*{\veca}{\bm{a}}
\newcommand*{\vecb}{\bm{b}}

% functions
\newcommand*{\One}{\mathbb{1}}
\newcommand*{\normtwo}[1]{\| #1 \|}

% set theory
\newcommand*{\union}{\cup}
\newcommand*{\inters}{\cap}
\newcommand*{\setdiff}{\setminus}

% topics commands
\newcommand*{\cls}{\text{cls}}
\newcommand*{\attr}{\text{attr}}
\newcommand*{\iou}{\text{IoU}}
\newcommand*{\bbox}{\bm{b}}
\newcommand*{\qj}{\bm{q}_j}
\newcommand*{\eqj}{e^{\qj}}
\newcommand*{\Eqj}{E^{\qj}}
\newcommand*{\ePI}{e^{\calP_{\bm{I}}}}
\newcommand*{\EPI}{E^{\calP_{\bm{I}}}}
\newcommand*{\fsim}{f_{\text{sim}}}
\newcommand*{\fagg}{f_{\text{aggregate}}}
\newcommand*{\fapplysim}{f_{\text{apply sim}}}
\newcommand*{\floss}{f_{\text{loss}}}
\newcommand*{\fdir}{f_{\text{dir}}}

\title{Weakly Supervised (?) Visual-Textual Grounding based on Concept Similarity}
\author{Luca Parolari}
\date{October 2021}

\begin{document}

\maketitle

\section{Introduction}

In this work we propose a simple and interpretable model that learns
from concept similarity. The goal is to solve phrase localization
problem under unsupervised settings. Given a noun phrase, we define
it's concept as the most important word in the phrase, while, given a
bounding box we define it's class as the label, among a dictionary of
labels, predicted with maximum confidence by an object detector for
that bounding box. The concept similarity is the similarity between
the phrase concept and bounding box class. Here, the key idea is that
the head of the phrase should be very similar (semantically speaking)
to the content of the bounding box and thus, to its class. Indeed, we
optimize the model to learn a representation in this similarity
subspace, where semantically similar features from image and text
should be close each other, otherwise they are forced to be
perpendicular. The unsupervision is obtained thought an oracle that,
based concept similarty, determines whether to attract or repulse
features.

% Our model is straightforward. First of all, we gather image features
% from the object detector and we compute five additional features
% representing bounding box top left and bottom right corners and area
% (spatial features). We then project those features to a subspace,
% performing a dimensionality reduction. Regarding the text, we fisrt
% compute the word embeddings using a pretrained word emebdding and then
% we encode the phrase throught an LSTM recurrent neural network. Our
% predictions are the results of the cosine similarity between projected
% image features and the last LSTM hidden neuron. In order to obtain the
% weak supervision, we compute concept similarity which is required to
% weakly supervise learning by attracting or repulsing features. The
% attraction is performed when concept similarity between two modalities
% is above a given threshold, otherwise repulsion is applied. Moreover,
% we accompany to concept similarity the attribute similarity which
% helps discriminate between same class bounding boxes. 

\section{Background}

\subsection{Phrase Localization}

Given in input an image $\bm{I}$ and a
sentence $\mathrm{S}$, phrase grounding consists in learning a mapping
$\gamma$ from the set $\calQ$ of noun phrases to a set of bounding box
$\calB$ defined on $\bm{I}$, i.e., $\gamma : \calI \times \calS
\rightarrow 2^{\calQ \times \calB}$ (Rigoni et al., 2021). So, given
an image $\bm{I}$ containing $e$ objects identified via the set of
bounding boxes $B_\calI = \{ b_i \}^e_{i=1}$, where $b_i \in \Rset^4$
is the vector of coordinates identifying a bounding box in $\bm{I}$,
and a sentence $\mathrm{S}$ containing $m$ noun phrases gathered in
the set $\calQ_S = \{ q_j \}^m_{j=1}$, where $q_j \in \Nset^2$ is a
vector containing as coordinates the initial and final character
positions in the sentence $\mathrm{S}$, $\gamma(\bm{I}, \mathrm{S})$
returns a set of couples $\{ (\bm{q}, \bm{b}) \mid \bm{q} \in
\calQ_{\mathrm{S}}, \bm{b} \in B_{\bm{I}} \}$ where each couple
$(\bm{q}, \bm{b})$ associates the noun phrase $\bm{q}$ to the bounding
box $\bm{b}$. Please, notice that the same noun phrase can be
associated to several different bounding boxes, as well as the same
bounding box can be associated to many different noun phrases.
Following the current literature, we assume that each noun phrase is
associated to one and only one bounding box. Bounding boxes, however,
can identify more objects, e.g. several persons in the case the noun
phrase is ``people''.

\end{document}
